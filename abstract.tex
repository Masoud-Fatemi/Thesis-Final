%*************************************************
% In this file the abstract is typeset.
% Make changes accordingly.
%*************************************************

\addcontentsline{toc}{section}{چکیده}
\newgeometry{left=2.5cm,right=3cm,top=3cm,bottom=2.5cm,includehead=false,headsep=1cm,footnotesep=.5cm}
\setcounter{page}{1}
\thispagestyle{empty}

~\vfill

\subsection*{چکیده}
\begin{small}
\baselineskip=0.7cm
امروزه با گسترش اینترنت و وب، انواع مختلف رسانه‌های اجتماعی نظیر وبلاگ‌ها و شبکه‌های اجتماعی به یک منبع بسیار عظیم از داده‌‌های متنی تبدیل شده‌اند. با پردازش این داده‌ها می‌‌توان اطلاعات سودمند و مفیدی در مورد مباحث مختلف، نظر افراد و احساس کلی‌ جامعه بدست آورد. از این جهت استفاده از مدل‌هایی که کاملا خودکار به تشخیص اطلاعات مفهومی‌ و احساس در اسناد متنی بپردازند بسیار مفید است. روش‌های مدل‌سازی موضوع، استخراج اطلاعات مفهمومی و همچنین تحلیل احساس از مهم‌ترین مباحث مطرح شده در زمینه‌ی پردازش زبان طبیعی و کاوش داده‌های متنی هستند. تمام مدل‌های مشترک موضوع و احساس ارائه شده بر پایه‌ی روش‌های آماری یا شبکه‌های بیزی هستند و تا کنون روشی برای مدل‌سازی این مسئله با استفاده از شبکه‌های عصبی ارائه نگردیده است. در این پایان‌نامه یک ساختار جدید برای مدل‌سازی مشترک احساس-موضوع در داده‌های متنی بر پایه‌ی شبکه‌‌ی عصبی ماشین بلتزمن محدود پیشنهاد می‌‌گردد. با تغییر ساختار این شبکه واضافه کردن یک لایه به آن که متناظر با احساس اسناد متنی است یک ساختار مولد احتمالی برای مدل‌سازی مشترک احساس و موضوع بر پایه‌ی ‌شبکه‌ي عصبی پیشنهاد می‌‌شود. رویکرد پیشنهاد شده یک رویکرد نظارت شده است که برای آموزش آن از الگوریم واگرایی مقابله استفاده می‌‌شود. لایه‌ی جدید اضافه شده در مدل پیشنهادی لایه‌ای با ماهیت توزیع احتمالی‌ چند جمله‌ای است که از آن می‌‌توان در فرآیند طبقه‌بندی اسناد متنی از نظر احساسی‌ یا دیگر کاربرد‌های نظارت شده استفاده کرد. مدل پیشنهادی در آزمایش‌هایی همانند: مدل‌سازی به عنوان یک مدل مولد، طبقه‌بندی احساس و بازیابی اطلاعات با مدل‌های موجود مقایسه گردید و نتایج بدست آمده توصیف ‌کننده‌ی قابلیت‌های مدل ارائه شده است. مشاهده گردید در فرآیند طبقه‌بندی احساس مدل پیشنهادی به طور میانگین ۱۱ درصد دقت بهتری نسبت به مدل پایه دارد. همچنین در فرآیند بازیابی اطلاعات بر روی پایگاه داده‌ی ۲۰ گروه خبری، رویکرد پیشنهادی با در نظر گرفتن احساس به طور متوسط ۹ درصد عملکرد بهتری نسبت به حالتی که در آن احساس در نظر گرفته نمی‌‌شود را به همراه دارد.\\

\texttt{}
\\
\noindent\textbf{کلمات کلیدی: مدل‌سازی موضوع، آنالیز احساس، شبکه‌ها‌ی عصبی، ماشین بلتزمن محدود، مدل احتمالاتی، الگوریتم واگرایی مقابله}
\end{small} 
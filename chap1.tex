% Chapter 1
\pagenumbering{arabic}
\setcounter{page}{2}
\chapter{مقدمه}
\label{chap1}
''آیا ماشین‌ها توانایی فکر کردن و یادگیری را دارند؟`` اولین بار این سوال در سال ۱۹۵۰ میلادی تحت عنوان یک مقاله با همین عنوان توسط ''آلن تورینگ`` 
که امروزه از او به عنوان پدر علم هوش مصنوعی\footnote{Artificial Intelligence}
یاد می‌‌گردد در یک مجله‌ی فلسفی‌ مطرح گردید. با گذر زمان و پیداش سیستم‌های کامپیوتری پیچیده و مطرح شدن مباحث مربوط به هوش مصنوعی به شکل امروزی، این سوال همواره به عنوان بزرگترین و چالش برانگیزترین سوال برای محققین و کارشناسان این حوزه مطرح بوده است. امروزه در تمام مباحث مربوط به هوش مصنوعی ما به دنبال روش‌ها، الگوریتم‌ها و ساختارهایی هستیم که بتوانند هرچه بهتر، به صورت خودکار و با دقت بالا یک  رفتار انسانی‌ و یا فرا انسانی‌ را با بیشترین سرعت ممکن انجام دهند. اعمالی مانند دسته‌بندی\footnote{Classification}،
 استخراج اطلاعات مفهومی‌\footnote{Subjective Information Extraction}،
  تحلیل\footnote{Analysis}
   و برچسب گذاری \footnote{Labeling}
   داده‌ها و از جمله فعالیت‌هایی‌ هستند که امروزه ما انجام بسیاری از آن‌ها را به ماشین‌ها واگذار می‌‌کنیم. 

در بین انواع مختلف داده شاید به جرات بتوان بیان کرد که داده‌های متنی\footnote{Text Data}
همواره  دارای سهم عظیمی‌ از نظر حجم و مقدار هستند. به خصوص با گسترش اینترنت  و وب در دهه‌ی اخیر با سرعتی‌ بسیار زیاد، انواع مختلف رسانه‌های اجتماعی نظیر وبلاگ‌ها، شبکه‌های اجتماعی و گروه‌های بحث در اینترنت به یک منبع بسیار عظیم و قوی از انواع مختلف داده و اطلاعات به  ویژه داده‌‌های  متنی تبدیل شده اند که با پردازش این داده‌ها می‌‌توان اطلاعات سودمند و مفیدی در مورد مباحث مختلف، نقطه نظر افراد و احساس کلی‌ جامعه بدست آورد
\cite{lin2012weakly}،
چرا که درک کردن و فهمیدن اینکه دیگر افراد چگونه فکر می‌‌کنند همواره یک هدف و بخشی بسیار مهم در بحث جمع‌آوری اطلاعات می‌‌باشد
\cite{pang2008opinion}.

در مباحث مربوط به حوزه هوش مصنوعی، فعالیت‌های انجام گرفته در زمینه کاوش داده‌ها\footnote{Data Mining}
به خصوص کاوش داده‌های متنی\footnote{Text Mining}
و همچنین پردازش زبان طبیعی\footnote{Natural Language Processing}،
 بیشتر از هر زمینه‌ی دیگری به تلاش برای درک و فهم این حجم عظیم از داده‌های متنی مربوط می‌‌شوند. حجمی عظیم از داده‌های متنی که بدون هیچ ساختار و قاعده و قانونی‌ می‌‌باشند و روز به روز مقدار آن‌ها با سرعت بسیاری چشمگیری در حال افزایش می‌‌باشد. در این میان وجود الگوریتم‌ها و روش‌هایی که بتوانند به صورت خودکار با این حجم بسیار زیاد از داده‌های بدون ساختار ارتباط برقرار کرده و اطلاعات مفید و سودمند را از آن برای ما استخراج کنند بیش از پیش احساس می‌‌گردد.

 		روش‌های مدل‌سازی موضوع\footnote{Topic Model}
 		و استخراج اطلاعات مفهمومی از داده‌های ورودی به خصوص داده‌های متنی و همچنین تشخیص احساس\footnote{Sentimen Detection}،
 		 همواره از مهمترین مباحث مطرح شده در زمینه‌ی پردازش زبان طبیعی و کاوش داده‌های متنی بوده است. این روش‌ها که اکثراً در دسته‌ی روش‌های بدون نظارت \footnote{Unsupervised}
 		 قرار می‌‌گیرند با اجرا بر روی یک پایگاه داده‌ از داده‌های متنی توانایی تشخیص و مدل‌سازی موضوعات و مفاهیم همراه با هر سند متنی را دارا می‌‌باشند. تشخیص احساس برای هر سند و هر موضوع در بحث بازیابی اطلاعات\footnote{Information Retrieval}
 (IR)
 		 نیز می‌‌تواند به اندازه تشخیص اطلاعات موجود در هر متن حائز اهمیت باشد. از این جهت داشتن مدل‌هایی که به صورت اتوماتیک و کاملا خودکار به مدل‌سازی موضوع و تشخیص اطلاعات مفهومی‌ و احساس در اسناد بپردازند می‌تواند بسیار مفید باشد.
 		
\section{اهداف پایان‌نامه}
کاری که ما در این پایان‌‌نامه قصد انجام آن را داریم نیز در همین راستا می‌‌باشد و هدف ارائه‌ی روشی‌ بر پایه‌ی شبکه‌های عصبی مصنوعی \footnote{Artificial Neural Networks}
برای استخراج موضوع های مختلف و احساسا‌ت همراه  با آن ها در یک مجموعه از داده‌های متنی می‌‌باشد. بیشتر کارهایی که در این زمینه وجود دارد بر پایه‌ی مدل‌های آماری و شبکه‌های بیزی \footnote{Bayesian Networks}
می‌‌باشند که دچار پیچیدگی محاسباتی می‌‌باشند. در بحث شبکه‌های عصبی مصنوعی بر خلاف مدل‌های آماری، روش‌های زیادی وجود ندارند که برای ما مدل کردن موضوع و احساس را انجام دهند. لذا مدلی‌ که در این پایان‌‌نامه پیشنهاد شده دارای رویکردی جدید برای یک مساله‌ی جدید می‌باشد که پس از پیاده‌سازی و آزمایش بر روی پایگاه داده‌های مختلف با مدل‌های موجود مقایسه می‌‌شود.


تمرکز ما در انجام این پایان‌‌نامه و روش پیشنهادی پردازش بر روی داده‌های متنی می‌‌باشد. در تقابل با داده‌های متنی هدف پیدا کردن توزیع موضوع‌های مختلف موجود در مجموعه اسناد پایگاه داده و همچنین توزیع کلمات و احساس همراه با هر موضوع با استفاده شبکه‌های عصبی مصنوعی می‌‌باشد. موضوع کلی و فرآیند مورد نظر در داده‌های متنی تحت عنوان مدل کردن موضوع شناخته می‌‌شود که در مباحث مربوط به هوش مصنوعی در دسته ی کارهای مربوط به یادگیری ماشین، پردازش زبان طبیعی، شبکه‌های عصبی مصنوعی و کاوش احساسات قرارمی‌‌گیرد.

\section{نوآوری‌های پایان‌نامه}
در بحث مدل‌سازی موضوع با استفاده از شبکه‌های عصبی در سال‌های اخیر تعداد اندکی‌ روش ارائه شده است. اما در زمینه‌‌ی مدل‌سازی مشترک احساس و موضوع با استفاده از شبکه‌های عصبی تا کنون هیچ مدلی‌ مطرح نشده و مورد آزمایش قرار نگرفته است. نتایج بهتر مدل‌های شبکه عصبی در بحث مدل‌سازی موضوع در مقایسه با روش‌های پیشین که از ساختارهای گرافی‌ و مدل‌های بیزی استفاده می‌‌کردند، همچنین عدم وجود روشی‌ برای تشخیص همزمان احساس و موضوع در داده‌های متنی با استفاده از شبکه‌های عصبی منجر به رویکرد پیشنهادی در این پژوهش برای مدل‌سازی مشترک احساس و موضوع در داده‌های متنی بر پایه‌ی شبکه‌های عصبی گردید.

\section{مروری بر فصل‌های پایان‌نامه}
در این تحقیق ابتدا در فصل دوم به معرفی‌ مفاهیم پایه می‌‌پردازیم. منظور از مفاهیم پایه ‌تعاریف و مفاهیم اولیه‌ و مورد نیاز در حوزه‌ی مدل‌های احتمالاتی، شبکه‌های عصبی و مدل‌سازی احساس و موضوع می‌‌باشند. در این فصل دو دسته‌ی مهم از مدل‌های احتمالاتی یعنی‌ مدل‌های مولد و افتراقی را تعریف می‌‌کنیم و بیان می‌کنیم که مدل پیشنهادی در این پایان‌‌نامه در کدام دسته‌ مدل‌ها قرار می‌گیرد. همچنین یک الگوریتم آموزش برای دسته‌ی خاصی‌ از شبکه‌های عصبی و همچنین توابع و مفاهیم پایه در بحث مدل‌سازی موضوع را به صورت کامل تعریف می‌کنیم.

در ادامه و در فصل سوم به مرور کارهای پیشین در زمینه‌ی تخمین توزیع‌های احتمالی‌ در داده‌های ورودی، مدل‌سازی احساس و مدل‌سازی احساس‌-موضوع در داده‌های متنی می‌‌پردازیم. ضمن تعریف مهم‌ترین مدل‌های موجود در این زمینه و بیان نقاط ضعف و قدرت آن‌ها توضیح می‌‌دهیم که ایده‌ی این پژوهش از کجا و به چه دلیل شکل گرفته و نسبت به مدل‌های پیشین از چه مزیت‌هایی برخوردار می‌‌باشد.

در فصل چهارم کلیات نظری و تئوری مدل پیشنهادی در این پژوهش به صورت دقیق بیان می‌‌شوند. با بیان کاستی‌های رویکردهای موجود در حوزه‌ی مدل‌سازی احساس و موضوع با استفاده از شبکه‌های عصبی یک روش جدید برای این منظور پیشنهاد می‌‌گردد. در این فصل با معرفی‌ یک مدل معروف به عنوان پایه و زیرساخت و چند روش گسترش یافته از آن ساختار مدل جدید تعریف و قسمت‌های مختلف آن شرح داده می‌‌شوند و روابط مورد نیاز برای هر بخش به صورت دقیق تعریف می‌‌شوند.

 در فصل پنجم مراحل شبیه‌سازی مدل پیشنهادی و نتایج حاصل از آن به طور مفصل توضیح داده خواهد شد.
 در ادامه و در فصل آخر، نتیجه‌گیری حاصل از انجام این پژوهش شرح داده خواهد شد. همچنین راهکارهایی برای بهبود و توسعه مدل پیشنهادی ارائه خواهد شد.
\thispagestyle{empty} 














\chapter{نتیجه‌گیری و پیشنهادات}
\thispagestyle{empty}
\section{نتیجه‌گیری}
پیشرفت گسترده و چشمگیر در دنیای امروز که از آن به عنوان عصر ارتباطات و تکنولوژی یاد می‌کنیم منجر به تولید حجم بسيار زیادی از شکل‌های مختلف داده شده است. داده‌های تصویری, صوتی, متنی و غیره که هر روز مقدار آن‌ها بخصوص در بستر اینترنت رو به افزایش است. استفاده از این حجم عظیم از انواع مختلف داده که با سرعتی بسيار زیاد در حال افزایش نیز هستند نیاز به رویکردها و ساختارهایی دارد که توانایی تطبیق با این نرخ رشد را داشته باشند. مدل‌هایی که با پردازش اتوماتیک و خودکار این داده‌ها بتوانند از آن‌ها اطلاعات مفید و مورد نظر را استخراج کنند. بدون شک این حجم گسترده از شکل‌های مختلف داده حاوی اطلاعات سودمند بسياری هستند که با استخراج آن‌ها می‌توان در کاربردهای گوناگون بهره‌های مناسبی از آن‌ها بدست آورد. 

داده‌های متنی نیز در این میان از سهم بالایی برخوردار هستند. این شکل از داده هر روز در شبکه‌های اجتماعی, سایت‌های خرید و فروش, گروه‌های بحث و تبادل نظر, مجلات برخط و غیره در حال تولید و افزایش هستند. با تحلیل این داده‌ها, پردازش واستخراج اطلاعات از آن‌ها می‌توان به نتایج جالبی از قبیل مهم‌ترین موارد مطرح شده در شبکه‌های اجتماعی,احساس کلی کاربران و افراد جامعه در موارد خاص, بیشترین و مهم‌ترین موضوع‌های مطرح شده در مجلات و برخظ و غیره بدست آورد. هدف ما در این پژوهش نیز در همین راستا قرار داشت. ما به دنبال ساختاری مناسب و رویکردی جدید برای مدل‌سازی اسناد متنی واستخراج اتوماتیک اطلاعات, بخصوص اطلاعات مفهومی واحساس موجود در اسناد بودیم.

با توجه به مطالب بیان شده, هدف کلی در این پژوهش ساخت مدلی بر پایه‌ی شبکه‌های عصبی برای مدل‌سازی مشترک موضوع واحساس در داده‌های متنی است. در فصل سوم این پژوهش با روش‌های پیشین موجود در این زمینه آشنا شدیم. همان‌طور که گفته شد ساختارهای موجود در اکثر موارد تنها به مدل‌سازی موضوع و یا تشخیص احساس در پایگاه داده می‌پردازند. بررسی‌های انجام شده نشان داد که در زمینه‌ی مدل‌سازی مشترک موضوع واحساس تنها دو مدل
ASUM (\ref{chap3sec4sub2})
و 
JST (\ref{chap3sec4sub1})
وجود دارند که با آن‌ها نیز به صورت کامل آشنا شدیم. گسترش شبکه‌های عصبی در سال‌های اخیر و استفاده‌ی فراوان از آن‌ها در بخش‌ها و زمینه‌های مختلف, همچنین عدم وجود ساختاری بر پایه‌ی شبکه‌های عصبی در زمینه‌ی مدل‌سازی مشترک احساس و موضوع به همراه کمبودها و کاستی‌های مدل‌های موجود در این زمینه دلایل اصلی نگارند‌ه‌ی این پژوهش برای ساخت رویکردی جدید در این زمینه و انجام این پژوهش بوده است.

در این پژوهش یک رویکرد نظارت شده با استفاده از شبکه‌های عصبی برای مدل‌سازی مشترک موضوع واحساس در داده‌های متنی پیشنهاد شد. این ساختار که در دسته روش‌های احتمالاتی مولد دسته‌بندی می‌گردد با گسترش مدل معروف ماشین بلتزمن محدود ایجاد می‌شود. در این رویکرد پیشنهادی یک لایه‌ی جدید با ماهیت توزیع احتمالاتی چندجمله‌ای به مدل اضافه شده و منجر به یادگیری ویژگی‌های بهتر و متمایز کننده‌تری برای هر سند در لایه‌ی مخفی
می‌شود. همان‌طور که در فصل 
\ref{chap5}
بیان گردید برای یادگیری و آموزش در این مدل از الگوریتم واگرایی مقابله استفاده می‌کنیم که یک روش تقریبی برای تخمین گرادیان می‌باشد.

برای ارزیابی مدل پیشنهادی از پایگاه داده‌های بازبینی فیلم
،(MR)
 20 گروه خبری 
(20NG)
 و احساس چند دامنه 
(MDS)
 که همگی از پایگاه داده‌های شاخص در بحث مدل‌سازی موضوع و احساس  در داده‌های متنی هستند، استفاده کردیم. با استفاده از یک معيار معروف برای ارزیابی مدل‌های مولد به نام سرگشتگی, رویکرد پیشنهادی در این پایان‌‌نامه را در فرآیند مدل‌سازی اسناد متنی ارزیابی کردیم. با توجه به نتایج بدست آمده در این بخش ادعا می‌کنیم که با در نظر گرفتن احساس موجود در اسناد و ایجاد ساختاری مانند آنچه که ما در این پژوهش انجام دادیم, یک مدل مولد بهتر برای مدل‌سازی اسناد ساخته می‌شود. همچنین با استفاده از پایگاه داده 
MR
 دقت رویکرد پیشنهادی را در زمینه‌های طبقه‌بندی احساس و ارزیابی موضوع‌های یاد گرفته شده مورد برررسی قرار دادیم و نتایج مربوط به آن را در بخش‌های
\ref{chap5sec8}
و
\ref{chap5sec9}
گزارش کردیم. در بخش
\ref{chap5sec10}
نیز رویکرد پیشنهادی در این پژوهش را در بحث بازیابی اطلاعات در مقایسه با مدل
RS
مورد ارزیابی قرار دادیم. با انجام آزمایش بر روی ۲ پایگاه داده‌‌ی مختلف و مقایسه‌ی نتایج مشاهده گردید که روش پیشنهادی نتایج بهتری را در بحث بازیابی اطلاعات بر روی داده‌های متنی دارد و از دقت بالاتری در این زمینه برخوردار است.

علی‌رغم اینکه در زمینه‌ی طبقه‌بندی احساس نتایج بدست آمده توسط رویکرد پیشنهادی در مقایسه با روش‌های موجود در بعضی حالت‌ها رقابتی است, اما مدل پیشنهادی در این پژوهش از نقاط قوت دیگری همچون نمایش موضوع‌های یاد گرفته شده و دقت بهتر در فرآیند مدل‌سازی به عنوان یک مدل مولد برخوردار است که آن را از دیگر روش‌های موجود در این زمینه متمایز می‌کند.


\section{پیشنهاد‌ها}
مدل پیشنهادی در این پژوهش
(فصل 
\ref{chap5})
از یک ساختار نظارت شده برای مدل‌سازی استفاده می‌کند. با توجه به اینکه حجم عظیمی از داده‌های موجود در بستر اینترنت داده‌های بدون برچسب هستند, لذا ساخت مدل‌های نیمه نظارتی, نظارت شده‌ی ضعیف و بدون نظارت به صورت قابل ملاحظه‌ای کاربردی و حائز اهمیت هستند. ساخت این دسته از مدل‌ها با گسترش رویکرد پیشنهادی از جمله مهم‌ترین کارهای پیش‌ رو در این پایان‌نامه می‌باشد. 

%در روش پیشنهادی نسبت اندازه لایه‌ی 
%$Sentiment$
%اندازه لایه‌ی
%$Visible$
%و
%$Hidden$
%مقدار بسیار کوچکی است که این امر می‌تواند موجب پایین آمدن کیفیت آموزش و نتایج نهایی گردد. می‌توان با تغییر رابطه‌ی محاسبه‌ی انرژی و به تبعیت از آن روابط مربوط به آپدیت پارامترها به گونه‌ای که مشکل عدم تعادل در اندازه‌ی لایه‌ها برطرف شود، به مقدار مناسب‌تری برای پارامترهای در پی فرایند آموزش رسید و همچنین نتایج بهتری در بخش عملی از مدل بدست آورد.

پیشنهاد دیگری که در این قسمت برای کارهای آینده می‌تواند مطرح گردد استفاده از اید‌ه‌ی ماشین بلتزمن محدود شرطی و ارائه‌ی مدل‌های عمیق\footnote{Deep Model}
برای استخراج ویژگی‌های سطح بالاتر است. در ساختار پیشنهادی در این پایان‌‌نامه به ازای هر سند یک بردار ویژگی به نام لایه‌ي مخفی
از آن استخراج می کنیم. پس ازاستخراج این بردار ویژگی از هر سند می‌توان خود این بردار را به عنوان یک داده‌ی ورودی برای یک شبکه عصبی 
RBM
در نظر گرفت و مجددا از آن ویژگی استخراج کرد. این ساختار منجر به یاد گرفتن ویژگی‌های متمایزتری برای هر داده‌ی ورودی می‌گردد.

رویکرد پیشنهادی در این پژوهش بر اساس کیسه‌ی کلمات عمل می‌کند. در استفاده از کیسه‌ی کلمات ساختار جملات و داده‌های ورودی از بین می‌روند و تنها تعداد تکرار کلمه‌ها مهم هستند. اما داده‌های متنی از سری داده‌های دنباله‌دار هستند و ترتیب در آن‌ها اهمیت دارد. لذا با در نظر گرفتن وابستگی کلمه‌ها در کنار یکدیگر و ترتیب آن‌ها می‌توان ویژگی‌های متمایز کننده‌تری از متن استخراج کرد. با توجه به اینکه در اکثر مواردی که بحث داده‌های دارای توالی مطرح می‌شود از شبکه‌های عصبی بازگشتی\footnote{Recurrent Neural Networks}
برای پياده‌سازی استفاده می‌گردد, لذا پیشنهاد می‌کنیم در این حالت نیز از این شبکه‌ها استفاده شود.

%\section{دسترسی به پیاده‌سازی‌های و پایگاه داده‌ها}
%همان‌طور که پیش از این بیان گردید، برای پياده‌سازی در این پژوهش از زبان برنامه‌نویسی پایتون 2.7 در محیط سیستم عامل لینوکس استفاده شده است. فایل‌های مربوط به پياده‌سازی‌های انجام شده بر روی سایت گیت‌هاب بارگذاری شده\footnote{https://github.com/Masoud-Fatemi/final.git}
%و قابل دسترسی و دانلود هستند. همچنین فایل‌های مربوط به پایگاه داده‌های استفاده شده در این پژوهش در یک فایل فشرده روی 
%DropBox
% قابل دریافت هستند\footnote{https://www.dropbox.com/s/yxgos056qgx92sq/DataSets.zip?dl=0}.
%

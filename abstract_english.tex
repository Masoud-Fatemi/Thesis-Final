
\addcontentsline{toc}{section}{چکیده انگلیسی}
\thispagestyle{empty}

\begin{latin}
	\begin{center}
		
		{\Huge \textbf{Extension of Restricted Boltzman Machine for Joint Sentiment Topic Modeling in Text Data}}
		
		\vspace{1cm}
		
		{\LARGE{Masoud Fatemi}}
		
		\vspace{0.2cm}
		
		{\small m.fatemi@ec.iut.ac.ir}
		
		\vspace{0.5cm}
		
		June , 2017
		
		\vspace{0.5cm}
		
		Department of Electrical and Computer Engineering
		
		\vspace{0.2cm}
		
		Isfahan University of Technology, Isfahan 84156-83111, Iran
		
		\vspace{0.2cm}
		
		Degree: M.Sc. \hspace*{3cm} Language: Farsi
		
		\vspace{1cm}
		
		{\small\textbf{Supervisor: Prof. Mehran Safayani (safayani@cc.iut.ac.ir)}}\\
		{\small\textbf{Advisor: Prof. Abdolreza Mirzaei (mirzaei@cc.iut.ac.ir)}}
	\end{center}
	~\vfill
	
	
	
	\noindent\textbf{Abstract}
	
	\begin{small}
		\baselineskip=0.6cm
Recently by the development of the internet and web, different types of social media such as web blogs become an immense source of text data. Through the processing of these data, it is possible to discover practical information about different topics, individual’s opinions and a thorough understanding of the society. Therefore, applying models which can automatically extract the subjective information from the documents would be efficient and helpful. Topic modeling methods, also sentiment analysis are the most raised topics in the natural language processing and text mining fields. Existed joint sentiment topic models are based on either statistical methods or bayesian networks and there are no solution for this problem using neural networks. In this thesis a new structure for joint sentiment-topic modeling based on Restricted Boltzman Machine (RBM) which is a type of neural networks is proposed. By modifying the structure of RBM as well as appending a layer which is analogous to sentiment of text data to it, we propose a generative structure for joint sentiment topic modeling based on neutral networks. The proposed method is supervised and trained by the Contrastive Divergence algorithm. The new attached layer in proposed model is a layer with multinomial probability distribution which can be used in text data sentiment classification or any other supervised application. The proposed model is compared with existing models. Experiment such as evaluating as a generative model, sentiment classification, information retrieval and the corresponding results demonstrate the efficiency of the method. It is observed in the sentiment classification task the proposed model has a better accuracy on average of 11\% in comparison with the baseline model. Additionally, in the information retrieval on 20 news group‘s dataset, the proposed model by considering sentiment has an average of 9\% improved accuracy in respect to the case in which sentiment is disregarded. 
	
		
	\end{small}
	
	\vspace{0.5 cm}
	
	
	\noindent \textbf{Key Words}: Topic Model, Sentiment Analysis, Neural Networks, Restricted Boltzman Machine, Probabilistic Model, Contrastive Divergence Algorithm
\end{latin}


\addcontentsline{toc}{section}{چکیده انگلیسی}
\thispagestyle{empty}

\begin{latin}
	\begin{center}
		
		{\Huge \textbf{Extension of Restricted Boltzman Machine for Joint Sentiment Topic Modeling in Text Data}}
		
		\vspace{1cm}
		
		{\LARGE{Masoud Fatemi}}
		
		\vspace{0.2cm}
		
		{\small m.fatemi@ec.iut.ac.ir}
		
		\vspace{0.5cm}
		
		June , 2017
		
		\vspace{0.5cm}
		
		Department of Electrical and Computer Engineering
		
		\vspace{0.2cm}
		
		Isfahan University of Technology, Isfahan 84156-83111, Iran
		
		\vspace{0.2cm}
		
		Degree: M.Sc. \hspace*{3cm} Language: Farsi
		
		\vspace{1cm}
		
		{\small\textbf{Supervisor: Prof. Mehran Safayani (safayani@cc.iut.ac.ir)}}\\
		{\small\textbf{Advisor: Prof. Abdolreza Mirzaei (mirzaei@cc.iut.ac.ir)}}
	\end{center}
	~\vfill
	
	
	
	\noindent\textbf{Abstract}
	
	\begin{small}
		\baselineskip=0.6cm
Recently by the development of the internet and web, different types of social media such as web blogs became an immense source of text data. Through the processing of these data, not only is it possible to discover practical information about different topics, but it is also feasible to know about individual’s opinions and gain a thorough understanding of the society sentiment. Therefore, having models which can automatically extract the subjective information from the documents would be efficient and helpful. Topic modeling procedures in addition data mining and also sentiment analysis are the most topics which considered in the natural language processing and text mining fields. Most of the existed models in these fields are based on the statistical methods and Bayesian networks so that there is no method for sentiment-topic modeling based on neural networks. Besides, the majority of existed frameworks have some constraint and difficulties such as computational complexity. This Thesis proposes a new structure for joint sentiment-topic modeling based on Restricted Boltzman Machine which is a type of neural networks. By altering the structure of Restricted Boltzman Machine plus appending a layer to it which is analogous to text data sentiment, feasibly we can propose a generative structure for joint sentiment topic modeling based on neutral network. Proposed method is a supervised procedure which is trained by the Contrastive Divergence algorithm. The new attached layer in proposed model is a layer with the multinomial probability distribution quiddity which can be used in text data sentiment classification or any other supervised application. The proposed model is compared to existed models After implementing on document modeling as a generative model, sentiment classification and information retrieval the corresponding results have been reported thoroughly. It is observed that in the sentiment classification experiment the proposed model has a better accuracy on average of 11\% in comparison with baseline model. Additionally, in the process of information retrieval on 20 news group‘s dataset, the proposed model by considering sentiment has an average of 9\% enhanced function rather than the case in which sentiment is disregarded. 
	
		
	\end{small}
	
	\vspace{0.5 cm}
	
	
	\noindent \textbf{Key Words}: Topic Model, Sentiment Analysis, Neural Networks, Restricted Boltzman Machine, Probabilistic Model, Contrastive Divergence Algorithm
\end{latin}
